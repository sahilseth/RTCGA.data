% !TeX root = RJwrapper.tex
\title{RTCGA.data - The Family Of R Packages Containing TCGA Data}
\author{by Marcin Kosinski, Przemyslaw Biecek}

\maketitle

\abstract{
The following article presents RTCGA.data: a family of R packages
containing The Cancer Genome Atlas Project (TCGA) data. The Cancer
Genome Atlas (TCGA) is a comprehensive and coordinated effort to
accelerate our understanding of the molecular basis of cancer through
the application of genome analysis technologies, including large-scale
genome sequencing (1). We provide TCGA data in few separate packages
that are hosted on one GitHub repository, what made those luxurious data
easier to possess and manage. We hope providing researchers with
comprehensive catalogs of the key genomic changes in many major types
and subtypes of cancer will support advances in developing more
effective ways to diagnose, treat and prevent cancer.
}

\subsection{Motivation}\label{motivation}

The Cancer Genome Atlas (TCGA) Data Portal provides a platform for
researchers to search, download, and analyze data sets generated by
TCGA. It contains clinical information, genomic characterization data,
and high level sequence analysis of the tumor genomes (1). The key is to
understand genomics to improve cancer care.

\subsection{Data origin}\label{data-origin}

Data from TCGA are available through Firehose Broad GDAC portal (2). One
can select cohort (assigned to the cancer type) and data type,
i.e.~clinical data, and download a \texttt{tar.gz} file with compressed
data.

The main disadvantages of such data download methodology are:

\begin{itemize}
\itemsep1pt\parskip0pt\parsep0pt
\item
  The downloaded files are compressed \texttt{tar.gz} files and not
  everyone manages to unpack such files.
\item
  If one requires to download datasets containing i.e.~information about
  genes' expressions for all available cohorts types (TCGA collected
  data for more than 30 various cancer types) one would have to go
  through click-to-download process many times, which is inconvienent
  and time-consuming.
\item
  Clinical datasets from TCGA project are not in a standard tidy data
  format, which is: one row for one observation and one column for one
  variable. They are transposed what makes work with those data
  burdensome. That becomes more onerous when one would like to
  investigate many clinical datasets.
\item
  Datasets containing information on gene's mutations are not in one
  easy-to-handle file. Every patient has it's own file, what for many
  potential researchers may be a impassable barrier. Just think about
  BRCA cohort (breast cancer) with more that 1000 various patients and
  more than 1200 files with patients' genes mutations information.
\end{itemize}

\subsection{RTCGA.data family data}\label{rtcga.data-family-data}

For reasons described in previous section we prepared selected datasets
from TCGA project in an easy to handle and process way and embed them in
5 separate R packages. All packages can be installed from GitHub by
evaluating the following code:

\begin{Schunk}
\begin{Sinput}
if (!require(devtools)) {
    install.packages("devtools")
    require(devtools)
}
install_github("mi2-warsaw/RTCGA.data", 
                subdir = paste0("RTCGA.", 
                                c("clinical", "rnaseq", "mutations", "cnv", "PANCAN12")
                               )
 )
\end{Sinput}
\end{Schunk}

One package, i.e. \texttt{RTCGA.clinical} can be installed with the
command

\begin{Schunk}
\begin{Sinput}
if (!require(devtools)) {
    install.packages("devtools")
    require(devtools)
}
install_github("mi2-warsaw/RTCGA.data", 
                subdir = "RTCGA.clinical") 
\end{Sinput}
\end{Schunk}

If you are using Windows, make sure you have rtools {[}3{]} installed on
your computer, before evaluating aboved commands.

RTCGA.data family contains 5 packages:

\begin{itemize}
\itemsep1pt\parskip0pt\parsep0pt
\item
  \texttt{RTCGA.clinical} package containing clinical datasets from
  TCGA. Each cohort contains one dataset prepared in a tidy format. Each
  row, marked with patients' barcode, corresponds to one patient.
\item
  \texttt{RTCGA.rnaseq} package containing genes' expressions datasets
  from TCGA. Each cohort contains one dataset with over 20 thousand of
  columns corresponding to genes' expression. Rows correspond to
  patients, that can be matched with patient's barcode.
\item
  \texttt{RTCGA.mutations} package containint genes' mutations datsets
  from TCGA. Each cohort contains one dataset with extra column
  specifying patient's barcode which enables to distinguish which rows
  correspond to which patient.
\item
  \texttt{RTCGA.cnv} package \textcolor{red}{ explanation needed.}
\item
  \texttt{RTCGA.PANCAN12} package \textcolor{red}{ explanation needed.}
\end{itemize}

More detailed information about datasets included in RTCGA.data family
are shown in Table \ref{data_details}

\tiny

\begin{table}[h!]

\begin{tabular}{lrrrrrr}
\toprule
Disease Name & Cohort & *.clinical & *.mutations & *.rnaseq & *.cnv & *.PANCAN12 \\
\toprule
Adrenocortical carcinoma &  ACC & \ & \ & \ & \ & \ \\
Bladder urothelial carcinoma &  BLCA& \ & \ & \ & \ & \ \\
Breast invasive carcinoma & BRCA    & \ & \ & \ & \ & \ \\
Cervical and endocervical cancers & CESC    & \ & \ & \ & \ & \ \\
Cholangiocarcinoma &    CHOL    & \ & \ & \ & \ & \ \\
Colon adenocarcinoma     & COAD & \ & \ & \ & \ & \ \\
Colorectal adenocarcinoma & COADREAD    & \ & \ & \ & \ & \ \\
Lymphoid Neoplasm Diffuse & DLBC    & \ & \ & \ & \ & \ \\
Esophageal carcinoma &  ESCA    & \ & \ & \ & \ & \ \\
FFPE Pilot Phase II  & FPPP & \ & \ & \ & \ & \ \\
Glioblastoma multiforme & GBM & \ & \ & \ & \ & \   \\
Glioma &    GBMLGG  & \ & \ & \ & \ & \ \\
Head and Neck squamous cell carcinoma & HNSC& \ & \ & \ & \ & \ \\
Kidney Chromophobe &    KICH    & \ & \ & \ & \ & \ \\
Pan-kidney cohort (KICH+KIRC+KIRP) &    KIPAN   & \ & \ & \ & \ & \ \\
Kidney renal clear cell carcinoma & KIRC    & \ & \ & \ & \ & \ \\
Kidney renal papillary cell carcinoma & KIRP & \ & \ & \ & \ & \    \\
Acute Myeloid Leukemia &    LAML    & \ & \ & \ & \ & \ \\
Brain Lower Grade Glioma &  LGG & \ & \ & \ & \ & \ \\
Liver hepatocellular carcinoma &    LIHC & \ & \ & \ & \ & \    \\
Lung adenocarcinoma &   LUAD    & \ & \ & \ & \ & \ \\
Lung squamous cell carcinoma &  LUSC & \ & \ & \ & \ & \    \\
Mesothelioma &  MESO    & \ & \ & \ & \ & \ \\
Ovarian serous cystadenocarcinoma & OV & \ & \ & \ & \ & \  \\
Pancreatic adenocarcinoma & PAAD    & \ & \ & \ & \ & \ \\
Pheochromocytoma and Paraganglioma &    PCPG & \ & \ & \ & \ & \    \\
Prostate adenocarcinoma &   PRAD & \ & \ & \ & \ & \    \\
Rectum adenocarcinoma & READ & \ & \ & \ & \ & \    \\
Sarcoma &   SARC    & \ & \ & \ & \ & \ \\
Skin Cutaneous Melanoma &   SKCM    & \ & \ & \ & \ & \ \\
Stomach adenocarcinoma &    STAD    & \ & \ & \ & \ & \ \\
Stomach and Esophageal carcinoma &  STES    & \ & \ & \ & \ & \ \\
Testicular Germ Cell Tumors &   TGCT    & \ & \ & \ & \ & \ \\
Thyroid carcinoma & THCA    & \ & \ & \ & \ & \ \\
Thymoma &   THYM    & \ & \ & \ & \ & \ \\
Uterine Corpus Endometrial Carcinoma &  UCEC & \ & \ & \ & \ & \    \\
Uterine Carcinosarcoma &    UCS & \ & \ & \ & \ & \ \\
Uveal Melanoma &    UVM & \ & \ & \ & \ & \ \\
\bottomrule
\end{tabular}

\end{table}

\normalsize

\subsection{How to work with RTCGA.data
family}\label{how-to-work-with-rtcga.data-family}

\subsection{Patient's barcode as a key to merge
data}\label{patients-barcode-as-a-key-to-merge-data}

\subsection{Applications examples}\label{applications-examples}

\begin{Schunk}
\begin{Sinput}
[1] http://cancergenome.nih.gov/

[2] http://gdac.broadinstitute.org/
   
[3] http://cran.r-project.org/bin/windows/Rtools/

\bibliography{RJreferences}
\end{Sinput}
\end{Schunk}

\address{
Marcin Kosinski\\
Warsaw University of Technology\\
Faculty of Mathematics and Information Science\\ Koszykowa 75, 00-662 Warsaw, Poland\\
}
\href{mailto:M.P.Kosinski@gmail.com}{\nolinkurl{M.P.Kosinski@gmail.com}}

\address{
Przemyslaw Biecek\\
Warsaw University\\
line 1\\ line 2\\
}
\href{mailto:Przemyslaw.Biecek@gmail.com}{\nolinkurl{Przemyslaw.Biecek@gmail.com}}

